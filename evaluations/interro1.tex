
\documentclass[11pt,a4paper]{article}
\usepackage[utf8]{inputenc}
\usepackage[T1]{fontenc}
\usepackage[french]{babel}
\usepackage[top=3cm, bottom=2cm, left=2cm, right=2cm]{geometry}
\usepackage{stmaryrd}
\usepackage{amsmath}
\usepackage{amsfonts}
\usepackage{amssymb}
\usepackage{mathrsfs}
\usepackage{amsthm}
\usepackage{layout}
\usepackage{fancyhdr}

\newtheorem*{thm}{Théorème}
\newtheorem{ex}{Exercice}
\newtheorem*{nota}{Notation}
\newtheorem*{rem}{Remarque}
\newtheorem*{rem2}{Remarques}
\newtheorem{de2}{Définition}
\newtheorem{pro2}[de2]{Propriété}
\newtheorem{thm2}[de2]{Théorème}

\setlength{\parindent}{0cm}
\setlength{\parskip}{1ex plus 0.5ex minus 0.2ex}
\newcommand{\hsp}{\hspace{20pt}}
\newcommand{\HRule}{\rule{\linewidth}{0.5mm}}

\usepackage{comment}


\title{}

\date{}
\begin{document}


\pagestyle{fancy}

\fancyhead{}
 \fancyfoot{}

 \lhead{L3 Mathématiques\\pour l'enseignement}
\chead{\textbf{ Algorithmes pour l'enseignement\\Compléments de géométrie}\\} 
 \rhead{ Université de Lorraine \\  }

\newcommand{\lb}{\llbracket}
\newcommand{\rb}{\rrbracket}
\newcommand{\N}{\mathbb{N}}
\newcommand{\Z}{\mathbb{Z}}
\newcommand{\R}{\mathbb{R}}
\newcommand{\C}{\mathbb{C}}



\newcommand{\md}[3]{#1\ \equiv \ #2 \! \! \! \! \! \pmod {#3} }
\newcommand{\nmd}[3]{#1 \not \equiv #2 \! \! \! \! \!  \pmod {#3} }
\newcommand{\mda}[3]{#1 \equiv #2 \! \!  \pmod {#3} }
\newcommand{\nmda}[3]{#1 \not \equiv #2 \! \! \pmod {#3} }
\newcommand{\mo}[2]{#1 \! \! \! \! \! \pmod #2 }
\newcommand{\moa}[2]{#1 \! \!  \pmod {#2} }

\renewcommand{\Re}{\operatorname{Re}}
\renewcommand{\Im}{\operatorname{Im}}

\thispagestyle{fancy}
\begin{center}
%    \HRule \\[0.6cm]
    { \huge \bfseries
    $ $\\
Interrogation du 30 septembre 2026
     \\ [.5cm] }
    \HRule \\[1cm]
\end{center}

\begin{center}
\textbf{Durée : 1 heure 30.\\ Dans toute l'interrogation, le plan euclidien $\mathcal P$ est muni d'un repère orthonormé direct auquel se rapportent les coordonnées et les affixes.}
\end{center}


\begin{ex}
Géométriquement, à quoi correspond l'application $f : \C\to\C, z\mapsto i\bar z$ ?
\end{ex}


\begin{ex}
Résoudre sur $\C$ l'équation $z^2=8-6i$.
\end{ex}


\begin{ex}
Résoudre sur $\C$ l'équation $|z+1|=|z|+1$.
\end{ex}

\begin{ex}
Soit $z \in \C$. Montrer:
\[
\frac{\left|\Re z\right| + \left|\Im z\right|}{\sqrt 2} \leq |z| \leq \left|\Re z\right| + \left|\Im z\right|
\]
Interpréter en termes de carrés et de cercles et faire une figure.
\end{ex}


%------------------------------
\begin{ex}
% nom : caractérisation des triangles équilatéraux
% source : ?
% tags : triangle équilatéral, angles, distances
On note comme d'habitude $j=e^{2i\pi/3}$. Soit $ABC$ un triangle.
\begin{enumerate}
\item Montrer : $ABC$ équilatéral direct 
$\iff  a+jb+j^2c=0 \iff  a-b = -j^2(c-b)$.
\item Montrer : 
$ABC \text{ équilatéral indirect}  \iff a+j^2b+jc=0$.
\item En déduire : $ABC \text{ équilatéral}  \iff a^2+b^2+c^2=ab+bc+ca$.
\end{enumerate}
\end{ex}


\begin{ex}
% source : Costantini
% source : https://fr.wikipedia.org/wiki/Th%C3%A9or%C3%A8me_de_Napol%C3%A9on
% nom : Théorème de Napoléon
% tags : triangle équilatéral, angles, distances
Soit $ABC$ un triangle direct. Soient $P, Q, R$ tels que $CBP$, $ACQ$ et $BAR$ soient des triangles équilatéraux directs. On note $U, V, W$ les centres de gravité respectifs de ces trois triangles équilatéraux. Montrer que $UVW$ est équilatéral, de même centre de gravité que $ABC$, en utilisant l'exercice précédent.
\end{ex}


\end{document}