
\documentclass[11pt,a4paper]{article}
\usepackage[utf8]{inputenc}
\usepackage[T1]{fontenc}
\usepackage[french]{babel}
\usepackage[top=3cm, bottom=2cm, left=2cm, right=2cm]{geometry}
\usepackage{stmaryrd}
\usepackage{amsmath}
\usepackage{amsfonts}
\usepackage{amssymb}
\usepackage{mathrsfs}
\usepackage{amsthm}
\usepackage{layout}
\usepackage{fancyhdr}
\usepackage{url}

\newtheorem*{thm}{Théorème}
\newtheorem{ex}{Exercice}
\newtheorem*{nota}{Notation}
\newtheorem*{rem}{Remarque}
\newtheorem*{rem2}{Remarques}
\newtheorem{de2}{Définition}
\newtheorem{pro2}[de2]{Propriété}
\newtheorem{thm2}[de2]{Théorème}

\setlength{\parindent}{0cm}
\setlength{\parskip}{1ex plus 0.5ex minus 0.2ex}
\newcommand{\hsp}{\hspace{20pt}}
\newcommand{\HRule}{\rule{\linewidth}{0.5mm}}

\usepackage{comment}

\usepackage{listings}
\usepackage{xcolor}

\definecolor{codeblue}{rgb}{0,0,0.9}
\definecolor{codegreen}{rgb}{0,0.6,0}
\definecolor{codegray}{rgb}{0.5,0.5,0.5}
\definecolor{codepurple}{rgb}{0.58,0,0.82}
\definecolor{backcolour}{rgb}{0.95,0.95,0.92}

\lstdefinestyle{mystyle}{
    backgroundcolor=\color{backcolour},   
    commentstyle=\color{codegreen},
    keywordstyle=\color{codeblue},
    %numberstyle=\tiny\color{codegray},
    stringstyle=\color{codepurple},
    basicstyle=\ttfamily\footnotesize,
    breakatwhitespace=false,         
    breaklines=true,                 
    captionpos=b,                    
    keepspaces=true,                 
    %numbers=left,                    
    %numbersep=5pt,                  
    showspaces=false,                
    showstringspaces=false,
    showtabs=false,                  
    tabsize=2
}

\lstset{style=mystyle}

\title{}

\date{}
\begin{document}


\pagestyle{fancy}

\fancyhead{}
 \fancyfoot{}

 \lhead{ 2024/2025 \\  L3 Mathématiques
}
\chead{\textbf{ Algorithmes pour l'enseignement}\\} 
 \rhead{ Université de Lorraine \\  }

\newcommand{\lb}{\llbracket}
\newcommand{\rb}{\rrbracket}
\newcommand{\N}{\mathbb{N}}
\newcommand{\Z}{\mathbb{Z}}
\newcommand{\R}{\mathbb{R}}




\newcommand{\md}[3]{#1\ \equiv \ #2 \! \! \! \! \! \pmod {#3} }
\newcommand{\nmd}[3]{#1 \not \equiv #2 \! \! \! \! \!  \pmod {#3} }
\newcommand{\mda}[3]{#1 \equiv #2 \! \!  \pmod {#3} }
\newcommand{\nmda}[3]{#1 \not \equiv #2 \! \! \pmod {#3} }
\newcommand{\mo}[2]{#1 \! \! \! \! \! \pmod #2 }
\newcommand{\moa}[2]{#1 \! \!  \pmod {#2} }

\thispagestyle{fancy}

\begin{center}
%    \HRule \\[0.6cm]
    { \huge \bfseries
Interrogation 3 (5 novembre)
     \\ [0cm] }
    \HRule \\[0.5cm]
\end{center}

\begin{center}
\textbf{durée : 1 heure}
\end{center}



Pour chacun des algorithmes, on justifiera avec soin : \begin{itemize}
\item[•] que l'algorithme termine (si une boucle «~tant que~» est utilisée).

\item[•] que l'algorithme renvoie bien le résultat demandé.
\end{itemize}

Une réponse non justifiée sera notée sur les trois quarts des points.

On peut utiliser les opérateurs \verb+//+ et \verb+%+ sur les entiers. On évitera d'utiliser des fonctionnalités trop avancées de python, surtout celles non vues en TD : on rédigera en pseudo-code avec des techniques élémentaires. En cas de doute, poser simplement la question.




\bigskip

%Pour les boucles «~pour~», on adoptera les conventions suivantes. Si $a,b\in \Z$, «~pour $i$ allant de $a$ à $b$~» signifie «~pour $i$ parcourant en croissant l'intervalle $\llbracket a,b\rrbracket$~». Lorsque $b<a$, cet intervalle est vide, donc aucune des instructions dans la boucle «~pour~» n'est effectuée (ce sera par exemple le cas si on écrit «~pour $i$ allant de $1$ à $n$~», avec $n=0$). On n'utilisera pas d'autre type de boucles «~pour~» (décroissantes, saut d'indice etc) : dans ces situations, on utilisera l'instruction «~tant que~».

%On pourra utiliser l'instruction «~pour $i$ allant en décroissant de $b$ à $a$~», qui signifie (si $b\geq a$) «~pour $i$ prenant successivement les valeurs $b$, $b-1$,$\ldots$, $a$~». Lorsque $a>b$, aucune instruction dans la boucle «~pour~» n'est alors effectuée.

Dans cette interrogation, \og complexité\fg{} signifie \og complexité en temps\fg, on ne demandera pas de complexité en espace. Les complexités sont entendues dans le pire des cas.


\begin{ex}(Limites, équivalents)\label{limites}
\begin{enumerate}
\item Déterminer la limite de $(3\sqrt[n]{2} - 2\sqrt[n]{3})^n$.
\item Déterminer un équivalent de $\left(1+\frac{\ln n}{n}\right)^n$.
\item Montrer que $\sum_{k=0}^n k! \sim n!$.
\end{enumerate}
\end{ex}

\begin{ex}\label{equiv_suite}
\begin{enumerate}
\item Montrer qu'une suite équivalente à $2^n$ est croissante à partir d'un certain rang.
\item Une suite équivalente à $n^2$ est-elle croissante à partir d'un certain rang ?
\end{enumerate}
\end{ex}

\begin{ex}\label{tribonacci}
La suite dite de \og Tribonacci\fg{} est la suite $(u_n)$ définie par $u_0=0$, $u_1 = 0$, $u_2 = 1$ et pour tout $n\geq 0$, $u_{n+3} = u_{n+2}+u_{n+1}+u_n$.
Dans cet exercice, on demande plusieurs implémentations d'une fonction recevant en entrée un entier naturel $n$ et retournant la valeur de $u_n$.
\begin{enumerate}
\item Écrire une implémentation récursive simple. (Sans boucle.)
\item Écrire une implémentation itérative, sans récursivité, de complexité linéaire en $n$.
\item Bonus : écrire une implémentation récursive, sans boucle, de complexité linéaire en $n$. L'idée est la même qu'à la question précédente, on pourra utiliser une deuxième fonction intermédiaire pour la récursivité à l'intérieur de la fonction principale.
\end{enumerate}
À chaque fois, on prouvera la terminaison, la correction et dans les deux derniers cas, la complexité.
Plus d'informations sur cette suite sur \url{https://oeis.org/A000073} ou Wikipédia.
\end{ex}

\newpage

\section*{Correction}

\subsection*{Correction de l'exercice \ref{limites}}

\begin{enumerate}
\item On a $\sqrt[n]{2} = e^{\frac{\ln 2}{n}} = 1+\frac{\ln 2}{n} + o\left(\frac{1}{n}\right)$. De même, $\sqrt[n]{3} =1+\frac{\ln 3}{n}  +  o\left(\frac{1}{n}\right)$. Donc : 
\[ 3\sqrt[n]{2}  - 2\sqrt[n]{3}  
= (3-2) + \frac{3\ln 2 - 2\ln 3}{n} +o\left(\frac{1}{n}\right)
= 1+ \frac{\ln(8/9)}{n} + o\left(\frac{1}{n}\right)
\]
On calcule la puissance $n$-ème avec l'exponentielle et le log de la même manière : 
\[
(3\sqrt[n]{2} - 2\sqrt[n]{3})^n
= \exp\left( n \ln \left( 1+ \frac{\ln(8/9)}{n} + o\left(\frac{1}{n}\right) \right) \right)
= \exp(\ln(8/9)+o(1) ) \to 8/9
\]
\item On a $u_n = exp\left(n\ln\left(1+\frac{\ln n}{n}\right)\right)$. Comme $\frac{\ln n}{n} \to 0$, on a $\ln\left(1+\frac{\ln n}{n}\right) = \frac{\ln n}{n} + O\left(\frac{\ln^2 n}{n^2}\right)$, donc $n\ln\left(1+\frac{\ln n}{n}\right) = \ln n + O\left(\frac{\ln^2 n}{n}\right)$. 
Finalement, $u_n = exp\left( \ln n + O\left(\frac{\ln^2 n}{n}\right) \right) \sim n$.\\
Remarque : on n'était pas loin d'avoir de façon précise le terme d'après, mais ce n'était pas demandé  : $u_n = n- \frac{\ln^2(n)}{2} + o(\cdots)$. 
Pour vous entraîner, vous pouvez calculer le troisième terme, qui devrait être $\frac{\ln^3(n)(3\ln n + 8)}{24n}$.
\item Posons $u_n = \sum_{k=0}^n k!$.
En majorant brutalement les $n-1$ premiers termes, on obtient :
\[ 
1 \leq \frac{u_n}{n!} = \sum_{k=0}^n\frac{k!}{n!} 
= \frac{0!}{n!} 
+ \frac{1!}{n!}
+ \cdots
+ \frac{1}{n(n-1)}
+ \frac{1}{n}
+ 1
\leq 
(n-1)\times \frac{1}{n(n-1)} + \frac{1}{n} + 1 \to 1\] 
\end{enumerate}

\subsection*{Correction de l'exercice \ref{equiv_suite}}

\begin{enumerate}
\item Comme $2^n$ n'est jamais nul, l'hypothèse équivaut à $u_n/2^n \to 1$. Voici deux rédactions.
%\emph{ Montrons que si $(u_n)$ n'est pas croissante à partir d'un certain rang, alors ceci contredit la convergence de $u_n/2^n$ vers $1$. L'idée est que si à un certain indice $u_n$ décroît, alors $u_n/2^n$ va diminuer trop fortement et donc s'éloigner trop fortement de $1$, qui devrait être sa limite.}

\textbf{Rédaction 1.} La suite $u_n$ est strictement positive  partir d'un certain rang, et à partir de ce rang-là,  on peut écrire $\frac{u_{n+1}}{u_n} = \frac{u_{n+1}}{2^{n+1}}\frac{2^{n+1}}{u_n} \to 2$. Ceci implique la croissance.

\textbf{Rédaction 2.} On revient à la définition de convergence. 
Soit $N\in\N$ tel que pour tout $n\geq N$, $\frac{u_n}{2^n} \in [2/3,4/3]$.
Supposons par l'absurde qu'il existe $n\geq N$ tel que $u_{n+1}<u_n$.
Alors, on a $\frac{u_{n+1}}{2^n} < \frac{u_n}{2^n} \leq \frac43$ et donc $\frac{u_{n+1}}{2^{n+1}} = \frac12\frac{u_{n+1}}{2^n}   < \frac46 = \frac23$.
C'est absurde car on doit avoir $\frac{u_{n+1}}{2^{n+1}} \in [2/3, 4/3]$.
\item Non. Considérons la suite $(u_n)$ dont les premières valeurs sont : 
$0,-1,4,4-1,16,16-1,36,36-1,64,64-1$ etc.

Elle est définie par :
\[ 
u_n = \begin{cases}
n^2 & \text{si $n$ est pair,}\\
(n-1)^2-1 & \text{sinon}
\end{cases}
\]
Pour $n>0$, notons $v_n := u_n/n^2$. Alors $v_n = 1$ si $n$ est pair, et si $n$ est impair, alors $v_n = \frac{(n-1)^2-1}{n^2} = 1-\frac{1}{n}$. Ceci montre que $v_n$ tend vers $1$ et donc que $u_n \sim n^2$.

%\emph{Question annexe à méditer : si $u_n$ est une suite qui tend vers $+\infty$, dans quels cas a-t-on $u_n \sim u_{n+1}$ ? Ce n'est pas le cas pour la suite $2^n$, mais c'est le cas pour $n^2$.}
\end{enumerate}

\subsection*{Correction de l'exercice \ref{tribonacci}}
\begin{enumerate}
Remarque : en conditions de TP, les fonctions pourraient être testée et chronométrées avec :
\begin{lstlisting}[language=Python]
def test_tribo():
	assert tribo(3) == 1
	assert tribo(4) == 2
	assert tribo(5) == 4
	assert tribo(9) == 44
	print("Test : tout semble ok")
	
import time
def perf_tribo(n):
	start = time.perf_counter()
	t = tribo(n)
	end = time.perf_counter()
	delta = end - start
	return delta
\end{lstlisting}

\item Implémentation récursive naïve : 
\begin{lstlisting}[language=Python]
def tribo(n):
	if n <= 1:
		return 0
	if n == 2:
		return 1
	return tribo(n-1)+tribo(n-2)+tribo(n-3)
\end{lstlisting}
Remarque : avec cette implémentation, tribo(30) met quelques seconde à s'exécuter sur ma machine (récente, 2020) et tribo(33) met presque seize secondes.\\
Questions à creuser pour s'entraîner aux TP : estimer la complexité de cette implémentation. Mesurer le temps d'execution de tribo(15), tribo(20), tribo(25), essayer d'estimer le temps d'execution de tribo(n) avec $n$ entre $30$ et $35$, si si le temps estimé semble raisonnable, vérifier. Essayer de déterminer expérimentalement la complexité, en supposant que c'est une suite géométrique en $n$.
% on pourrait grapher les temps, puis chercher l'exposant
% en théorie la complexité est en \tau^n, avec \tau\sim 1,84
% https://fr.wikipedia.org/wiki/Suite_de_Tribonacci
\item Implémentation itérative qui s'exécute en $O(n)$ :\\
Premier essai : on calcule la liste de toutes les valeurs jusqu'à $n$, que l'on stocke dans une liste et on retourne la dernière valeur de cette liste :
\begin{lstlisting}[language=Python]
def tribo(n):
	t =  [0,0,1]
	if n <= 2:
		return t[n]
		
	for i in range(n-2):
		t.append(t[i]+t[i+1]+t[i+2]) # plus clair (?) : t[-3]+t[-2]+t[-1]
	return t[-1] # ou t.pop(), ou t[len(t)-1]
\end{lstlisting}
On peut comparer la vitesse d'exécution avec la version précédente : le calcul de tribo(1000), et même de tribo(10000), est quasiment instantané. Celui de tribo(100000) prend trois secondes. Enfin, le calcul de tribo(1000000) a semblé prendre bien plus de 30 secondes et a été abandonné. 
À noter que le stockage de toutes les valeurs de la suite est évidemment susceptible de ralentir l'exécution, ce qui motive une modification.

Amélioration : on ne garde en mémoire que les trois dernières valeurs à chaque fois. Ceci utilise beaucoup moins de mémoire. (Complexité spatiale en $O(1)$ au lieu de $O(n)$.)
\begin{lstlisting}[language=Python]
def tribo(n):
	t =  [0,0,1]
	if n <= 2:
		return t[n]
		
	for i in range(n-2):
		somme  = t[0]+t[1]+t[2]
		t[0] = t[1]
		t[1] = t[2]
		t[2] = somme
	return t[2]
\end{lstlisting}

Ou, plus court en utilisant les possibilités d'affectations multiples de Python:
\begin{lstlisting}[language=Python]
def tribo(n):
	t =  [0,0,1]
	if n <= 2:
		return t[n]
		
	for i in range(n-2):
		t[0], t[1], t[2] = t[1], t[2], t[0]+t[1]+t[2]
		
	return t[2]
\end{lstlisting}

Remarque : toujours sur la même machine,  le calcul de tribo(1000000) s'est exécuté en moins d'une minute. Le processus a utilisé 20 Go de mémoire, suite à quoi sortir de Python a pris quasiment une minute...


\item Implémentation récursive de complexité linéaire: on va  utiliser une fonction récursive auxiliaire qui prend en entrée un tableau contenant les trois dernières valeurs calculées.
\begin{lstlisting}[language=Python]
def tribo_aux(t,n):
	if n == 2:
		return t[2]
	t[0], t[1], t[2] = t[1], t[2], t[0]+t[1]+t[2]
	return tribo_aux(t,n-1)
	
def tribo(n):
	if n <= 1:
		return 0
	return tribo_aux([0,0,1],n)
	
\end{lstlisting}
Noter que par défaut, Python limite de toute façon le nombre d'appels récursifs.
\end{enumerate}


\end{document}
