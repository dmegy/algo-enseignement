
\documentclass[11pt,a4paper]{article}
\usepackage[utf8]{inputenc}
\usepackage[T1]{fontenc}
\usepackage[french]{babel}
\usepackage[top=3cm, bottom=2cm, left=2cm, right=2cm]{geometry}
\usepackage{stmaryrd}
\usepackage{amsmath}
\usepackage{amsfonts}
\usepackage{amssymb}
\usepackage{mathrsfs}
\usepackage{amsthm}
\usepackage{layout}
\usepackage{fancyhdr}

\newtheorem*{thm}{Théorème}
\newtheorem{ex}{Exercice}
\newtheorem*{nota}{Notation}
\newtheorem*{rem}{Remarque}
\newtheorem*{rem2}{Remarques}
\newtheorem{de2}{Définition}
\newtheorem{pro2}[de2]{Propriété}
\newtheorem{thm2}[de2]{Théorème}

\setlength{\parindent}{0cm}
\setlength{\parskip}{1ex plus 0.5ex minus 0.2ex}
\newcommand{\hsp}{\hspace{20pt}}
\newcommand{\HRule}{\rule{\linewidth}{0.5mm}}

\usepackage{comment}

\usepackage{listings}
\usepackage{xcolor}

\definecolor{codeblue}{rgb}{0,0,0.9}
\definecolor{codegreen}{rgb}{0,0.6,0}
\definecolor{codegray}{rgb}{0.5,0.5,0.5}
\definecolor{codepurple}{rgb}{0.58,0,0.82}
\definecolor{backcolour}{rgb}{0.95,0.95,0.92}

\lstdefinestyle{mystyle}{
    backgroundcolor=\color{backcolour},   
    commentstyle=\color{codegreen},
    keywordstyle=\color{codeblue},
    %numberstyle=\tiny\color{codegray},
    stringstyle=\color{codepurple},
    basicstyle=\ttfamily\footnotesize,
    breakatwhitespace=false,         
    breaklines=true,                 
    captionpos=b,                    
    keepspaces=true,                 
    %numbers=left,                    
    %numbersep=5pt,                  
    showspaces=false,                
    showstringspaces=false,
    showtabs=false,                  
    tabsize=2
}

\lstset{style=mystyle}

\title{}

\date{}
\begin{document}


\pagestyle{fancy}

\fancyhead{}
 \fancyfoot{}

 \lhead{ 2024/2025 \\  L3 Mathématiques
}
\chead{\textbf{ Algorithmes pour l'enseignement}\\} 
 \rhead{ Université de Lorraine \\  }

\newcommand{\lb}{\llbracket}
\newcommand{\rb}{\rrbracket}
\newcommand{\N}{\mathbb{N}}
\newcommand{\Z}{\mathbb{Z}}
\newcommand{\R}{\mathbb{R}}




\newcommand{\md}[3]{#1\ \equiv \ #2 \! \! \! \! \! \pmod {#3} }
\newcommand{\nmd}[3]{#1 \not \equiv #2 \! \! \! \! \!  \pmod {#3} }
\newcommand{\mda}[3]{#1 \equiv #2 \! \!  \pmod {#3} }
\newcommand{\nmda}[3]{#1 \not \equiv #2 \! \! \pmod {#3} }
\newcommand{\mo}[2]{#1 \! \! \! \! \! \pmod #2 }
\newcommand{\moa}[2]{#1 \! \!  \pmod {#2} }

\thispagestyle{fancy}

\begin{center}
%    \HRule \\[0.6cm]
    { \huge \bfseries
Interrogation du 15 octobre 2024
     \\ [0cm] }
    \HRule \\[0.5cm]
\end{center}

\begin{center}
\textbf{durée : 1 heure 30}
\end{center}



Pour chacun des algorithmes, on justifiera avec soin : \begin{itemize}
\item[•] que l'algorithme termine (si une boucle «~tant que~» est utilisée).

\item[•] que l'algorithme renvoie bien le résultat demandé.
\end{itemize}

Une réponse non justifiée sera notée sur les trois quarts des points.

On peut utiliser les opérateurs \verb+//+ et \verb+%+ sur les entiers. On évitera d'utiliser des fonctionnalités trop avancées de python, surtout celles non vues en TD : on rédigera en pseudo-code avec des techniques élémentaires. En cas de doute, poser simplement la question.




\bigskip

Pour les boucles «~pour~», on adoptera les conventions suivantes. Si $a,b\in \Z$, «~pour $i$ allant de $a$ à $b$~» signifie «~pour $i$ parcourant en croissant l'intervalle $\llbracket a,b\rrbracket$~». Lorsque $b<a$, cet intervalle est vide, donc aucune des instructions dans la boucle «~pour~» n'est effectuée (ce sera par exemple le cas si on écrit «~pour $i$ allant de $1$ à $n$~», avec $n=0$). On n'utilisera pas d'autre type de boucles «~pour~» (décroissantes, saut d'indice etc) : dans ces situations, on utilisera l'instruction «~tant que~».

%On pourra utiliser l'instruction «~pour $i$ allant en décroissant de $b$ à $a$~», qui signifie (si $b\geq a$) «~pour $i$ prenant successivement les valeurs $b$, $b-1$,$\ldots$, $a$~». Lorsque $a>b$, aucune instruction dans la boucle «~pour~» n'est alors effectuée.

Dans cette interrogation, \og complexité\fg{} signifie \og complexité en temps\fg, on ne demandera pas de complexité en espace. Les complexités sont entendues dans le pire des cas.

\begin{ex}
Écrire une fonction \verb+expo+ qui prend en entrée un réel $a$ et un entier naturel $n$ et qui renvoie $a^n$, en n'utilisant que des sommes et des produits (et pas l'exponentiation).

On donnera une fonction récursive et aussi itérative, et on prouvera à chaque fois la correction de l'algorithme utilisé.

Ne pas chercher à faire de l'exponentiation rapide, c'est l'objet d'un exercice ultérieur. Utiliser l'algorithme naïf.
\end{ex}

\begin{ex}
Écrire une fonction \verb+evalPoly+ qui prend en entrée un polynôme $P$ (donné par la liste de ses coefficients, non vide par hypothèse) ainsi qu'un réel $a$, et qui retourne la valeur $P(a)$. Le polynôme sera entré sous forme de liste de coefficients, en commençant par le coefficient constant. Par exemple, si le polynôme est $3X^5+X^4+2X-1$, la liste des coefficients sera \verb+[-1,2,0,0,1,3]+. 

Consignes : écrire un algorithme naïf, sans chercher à optimiser. On peut utiliser la fonction \verb+expo+ de la question précédente.

Prouver la correction de l'algorithme proposé.
Quelle est la complexité de l'algorithme proposé, en fonction de la longueur de la liste des coefficients du polynôme ?
\end{ex}

Dans la suite, on améliore progressivement cet algorithme d'évaluation polynomiale.

\begin{ex}
Que fait l'algorithme suivant ($a\in\R$ et $n\in\N$) et quelle est sa complexité en fonction de $n$ ? 
\begin{lstlisting}[language=Python]
def f(a,n):
	if n == 0:
		return 1
	if n % 2 == 0 :
		return f(a*a, n//2)
	return f(a*a, n//2) * a
\end{lstlisting}
\end{ex}

\begin{ex}
Écrire une fonction non récursive qui fait la même chose que la précédente, avec la même complexité en $n$. 
(Il s'agit donc de remplacer le mécanisme de récursivité par une boucle \og tant que\fg.)

\end{ex}

\begin{ex}
En utilisant la fonction du deuxième exercice, proposer une amélioration \verb+evalPoly2+ de la fonction \verb+evalPoly+ pour évaluer un polynôme.
Les entrées et sorties sont les mêmes qu'à l'exercice 1, la complexité doit être meilleure.
Quelle est la complexité de votre algorithme ?
\end{ex}

Dans l'exercice suivant, on améliore encore l'algorithme.

\begin{ex}
L'algorithme de \textbf{Hörner} pour évaluer des polynômes consiste à ne pas recalculer les puissances à chaque fois depuis le début, mais à garder le mémoire les puissances successives pour calculer les suivantes. 
Typiquement, si l'on veut évaluer $X^4+2X^3$ au réel $a$, on va d'abord calculer $a^3$ puis multiplier par deux pour évaluer le premier monôme, mais il serait dommage de recommencer le calcul de $a^4$ puisqu'on a déjà $a^3$ : il suffit de multiplier par $a$, ce qui ne fait qu'une opération.
Si $P = \sum_{k=0}^d c_kX^k$ et $a\in\R$, l'algorithme de Hörner consiste à écrire :
\[ P(a) = \left(((c_d\times a+c_{d-1})\times a +\cdots )\times a +c_2)\times a+c_1 \right)\times a+c_0\]
et à effectuer le calcul en commençant par la parenthèse la plus \og profonde\fg.
Exemple : pour évaluer $X^3+3X^2+X+5$ en $a$, on calcule $a+3$, on multiplie le résultat par $a$ et on ajoute $1$ au résultat, ensuite on multiplie tout ceci par $a$ et on rajoute $5$.

\begin{enumerate}
\item Écrire une fonction \verb+evalPolyHorner+ qui implémente cet algorithme. (Toujours avec les mêmes entrées et sorties qu'à l'exercice 2.)
\item Quelle est la complexité de cet algorithme ? (Toujours en fonction du degré du polynôme, c'est-à-dire de la longueur de la liste de ses coefficients.
\end{enumerate}

\end{ex}


\newpage

{\Huge Attention, corrigé en cours de rédaction}

\paragraph{Correction de l'exercice 1 }$ $
Algorithme récursif : 
\begin{lstlisting}[language=Python]
def expo(a,n):
	if n == 0:
		return 1
	return a * expo(a,n-1)
\end{lstlisting}
Terminaison : la variable $n$ passée en argument dans l'appel récursif diminue de un à chaque appel, en temps fini elle vaut $0$, étape à laquelle les appels s'arrêtent.\\
Terminaison, version plus détaillée : on prouve que l'appel à \verb+exp(a,n)+ effectue exactement $n+1$ appels en tout. Preuve : par récurrence : \verb+exp(a,0)+ effectue exactement un appel, le premier.  Ensuite, l'appel de \verb|exp(a,n+1)| effectue un autre appel à \verb+exp(a,n)+, ce qui fait $1+(n+1) = n+2$ appels d'après l'hypothèse de récurrence. \\
Validité : On fixe $a$. Prouvons par récurrence sur $n \in \N$ que \verb+expo(a,n)+ retourne bien $a^n$. Par définition de la fonction, c'est vrai pour $n=0$. Montrons l'hérédité. Soit $n\in\N$, supposons que \verb+expo(a,n)+ retourne bien $a^n$ et montrons que  \verb|expo(a,n+1)| retourne bien $a^{n+1}$. Par définition de la fonction, comme $n+1 \neq 0$, \verb|expo(a,n+1)|  retourne \verb+a * expo(a,n)+. D'après l'hypothèse de récurrence,  ceci est égal à  $a\times a^n = a^{n+1}$.\\

Algorithme itératif :

\begin{lstlisting}[language=Python]
def expo(a,n):
	r = 1
	for k in range(1,n+1):
		r = r * a
	return r
\end{lstlisting}
Terminaison : la boucle \textit{for} tourne exactement $n$ fois.\\
Validité : Pour $n=0$, la fonction renvoie bien $1$. Montrons la correction pour $n\geq 1$. Pour $k$ entre $1$ et $n$, notons $r_k$ la valeur de la variable \verb+r+ à la fin de la boucle d'indice $k$.\\
Montrons par récurrence sur $k$ que pour tout $k$ entre $1$ et $n$, $r_k = a^k$.\\
Pour $k=1$ c'est vrai.\\
Soit $1\leq k \leq n-1$, supposons $r_k = a^k$ et montrons $r_{k+1} = a^{k+1}$. Dans la boucle d'indice $k+1$, l'affectation se traduit par $r_{k+1} = ar_k$. Par hypothèse de récurrence, $r_k=a^k$, donc on a bien $r_k=a^{k+1}$.\\
Fin de la preuve de validité : en sortie de boucle, c'est-à-dire à la fin de la boucle d'indice $n$, on a bien $r_n = a^n$.


\paragraph{Correction de l'exercice 2 }$ $

\begin{lstlisting}[language=Python]
def evalPoly(L,a):
	d=len(L)-1
	r=0
	for k in range(d+1):
		r += L[k] * expo(a,k)
	return r
\end{lstlisting}
Terminaison : on a une boucle for qui tourne un nombre prédéterminé de fois, et qui contient une fonction \verb+expo+ qui termine.\\

Validité : Notons $r_k$ la valeur de $r$ à la fin de la boucle d'indice $k$.
Alors, on a $r_k = \sum_{j=0}^k c_ja^j$. Preuve : récurrence sur $k$.\\
En fin de boucle, la variable \verb+r+ vaut donc $r_d = \sum_{j=0}^d c_ja^j = P(a)$.\\

Complexité : la fonction \verb+expo(a,k)+ est de complexité $O(k)$, donc $O(n)$ en majorant.
La fonction \verb+evalPoly+ est de complexité $O(n^2)$ avec $n$ la longueur de la liste.\\


\paragraph{Correction de l'exercice 3 }\par
C'est exponentiation rapide, avec une implémentation légèrement différente de celle vue en cours.\\
\textbf{Terminaison} : Pour tout entier $n$, la suite d'entiers $(u_k)$ définie par $u_0=n$ et la relation de récurrence $u_{k+1} = u_k // 2$ est une suite qui décroit strictement jusqu'au moment où elle stationne à zéro. L'algorithme termine donc.\\
Terminaison, version plus précise : si $u_k\geq 2$, alors par définition de la division euclidienne, $u_{k+1}$ est le nombre dont l'écriture en base deux est obtenue à partie de celle de $u_k$ en enlevant le chiffre des unités. Ce nombre a exactement un chiffre de moins que $u_k$ en base deux. Sinon, c'est-à-dire si $u_k<2$, alors $u_{k+1}=0$ et la récursivité s'arrête à l'étape d'après. Si $\ell$ est le nombre de chiffres en base deux de $n\neq 0$, alors \verb+f(a,n)+ provoque exactement $\ell$ autres appels récursifs à la fonction. Par exemple \verb+f(a,10)+ provoque quatre autres appels.\\
\textbf{Validité} : montrons que la fonction \verb+f(a,n)+ retourne $a^n$. Pour $n\in\N$, on note $P(n)$ l'assertion : \og pour tout\footnote{attention ici il est important de mettre le pour tout $a$ dans l'assertion à prouver, car on va appliquer l'hypothèse de récurrence sur une autre valeur de $a$.} réel $a$, la fonction \verb+f(a,n)+ retourne $a^n$.\fg Montrons maintenant par récurrence forte l'assertion $\forall n\in\N, P(n)$.\\
Initialisation : $P(0)$ est vraie par définition de la fonction.\\
Hérédité : soit $n\in \N$, supposons $P(k)$ vraie pour tout $j\leq n$ et montrons $P(n+1)$. Soit donc $a\in \R$. Comme $n+1>0$, la première condition est sautée. Si $n+1$ est pair, la quantité $(n+1)//2$ vaut donc $(n+1)/2$. La fonction retourne $f(a^2,(n+1)/2)$. Comme $(n+1)/2\leq n$, l'hypothèse de récurrence forte implique que $f(a^2,(n+1)/2) = (a^2)^{(n+1)/2} = a^{n+1}$.
Si par contre $n+1$ est impair, alors $(n+1)//2 =n/2\leq n$, et toujours par hypothèse de récurrence forte, on a $a\times f(a^2,n/2) = a\times(a^2)^{n/2} = a\times a^n=a^{n+1}$.\\
\textbf{Complexité} : Étant donné un exposant d'entrée $n$, notons $p$ le nombre d'appels à la fonction qui vont être effectués, y compris le premier appel.
À l'intérieur de la fonction, hors récursivité, la complexité est en $O(1)$ (entre une et deux comparaisons, moins de deux produits, moins d'une division).
La complexité de l'appel total, récursivité comprise, est donc en $O(p)$. Comme $p  = O(\log_2 n)$, on en déduit que la complexité totale est en $O(\log_2 n)$.\\
Remarque : comme dans la preuve précise de terminaison, on peut calculer $p$ en fonction de $n$ : c'est un de plus que le nombre de chiffres en base deux de $n$, si $n>0$. (Par exemple, si $n=2^k$, alors $p=k+2$.) 


\paragraph{Correction de l'exercice 4 }$ $

Proposition d'algorithme itératif d'exponentiation plus rapide que l'algorithme naïf: 
\begin{lstlisting}[language=Python]
def f(a,n):
	if n == 0:
		return 1
	r=1
	while n >= 1:
		k=1
		b=a
		while 2 * k <= n:
			k = 2 * k
			b = b * b
		r = r * b
		n = n - k
	return r
\end{lstlisting}

\textbf{Preuve de terminaison} : la boucle la plus profonde termine car la suite $2^k$ est non bornée.
Après l'arrêt de la boucle while la plus profonde, l'instruction \verb+n = n-k+ fait décroitre strictement $n$ car $k\geq 1$. Ceci montre que la boucle extérieure s'arrête en temps fini.\\
Terminaison, version plus précise : le nombre de chiffres de $n$ en base deux décroit strictement à chaque passage dans la boucle extérieure.\\

Preuve de validité : notons $n_0$ et $r_0$ les valeurs initiales de $n$ et $r$, avant la boucle while. Notons $n_i$ et $r_i$ les valeurs de $n$ et $r$ à la fin du $i$-ème passage dans la boucle while extérieure.
Alors, on a toujours, en fin de boucle extérieure : $a^{n_0} = r_i \times a^{n_i}$. (Récurrence sur $i$.)
En fin de boucle, on a $n_i = 0$ et donc $r_i= a^{n_0}$, qui est la valeur retournée.

Plus malin, avec une seule boucle while, en mettant en mémoire les exposants à rajouter dans le cas impair : 
\begin{lstlisting}[language=Python]
def f(a,n):
	if n == 0:
		return 1
	b = a
	c = 1
	while n > 1:
		if n % 2 == 1:
			n = n-1
			c = c * b
		else:
			n = n / 2
			b = b * b
	return b * c
\end{lstlisting}

Exercice : comprendre le fonctionnement de cette fonction, prouver la terminaison, la correction, comparer la complexité avec la proposition antérieure.



\paragraph{Correction de l'exercice 5 }$ $
On fait tout simplement , en notant \verb+exporapide+ une fonction d'exponentiation rapide comme celle donnée plus haut : 

\begin{lstlisting}[language=Python]
def evalPoly(L,a):
	d=len(L)-1
	r=0
	for k in range(d+1):
		r += L[k] * exporapide(a,k)
	return r
\end{lstlisting}

Estimation de la complexité :\\
On note $d$ le degré du polynôme en entrée et on calcule la complexité en fonction de $d$.
À l'intérieur de la boucle, la complexité de \verb+exporapide(a,k)+ est en $O(\log k)$. Comme $k\leq d$, ceci est en $O(\log d)$.

Chacun des $d+1$ appels est donc en $O(\log d)$. 
La complexité totale est en $O(d\log d)$.

\paragraph{Correction de l'exercice 6}$ $
\begin{lstlisting}[language=Python]
def evalPolyHorner(L,a):
	d = len(L)-1
	r = L[d]
	k = d
	while k >= 1:
		r = r * a + L[k-1]
		k = k-1
	return r
\end{lstlisting}

Preuve de terminaison : la variable $k$ décroit de $1$, elle finit donc par être nulle, moment où la boucle s'arrête. On pourrait aussi réécrire l'algorithme avec une boucle \og for\fg, d'ailleurs.

Preuve de validité : notons $P = \sum_{k=0}^d c_kX^k$ le polynôme.
Pour $k$ entre $d$ et zéro, notons $r_k$ la valeur de la variable $r$ juste avant le test d'entrée de la boucle d'indice $k$. 
On a $r_d = c_d$, $r_{d-1} = ac_d+c_{d-1}$, $r_{d-2} = a^2c_d + ac_{d-1}+c_{d-2}$.
Plus généralement, pour $k \in \llbracket 0,D\rrbracket$, notons $A(k)$ l'assertion «$r_k = \sum_{j=k}^{d}c_ja^{j-k}$».
Montrons par récurrence descendante sur $k \in \llbracket d,0\rrbracket$ que $k \in \llbracket 0,D\rrbracket, A(k)$.\\
Initialisation : $A(d)$ est vraie.\\
Soit $k \in \llbracket 1,d\rrbracket$. Supposons $A(k)$. Montrons $A(k-1)$.\\ 
On a donc $r_k = \sum_{j=k}^{d}c_ja^{j-k}$ et lors du passage dans la boucle, on a :
\[
r_{k-1} = ar_k+c_{k-1} 
= a\sum_{j=k}^{d}c_ja^{j-k} + c_{k-1} 
= \sum_{j=k}^{d}c_ka^{j-(k-1)} + c_{k-1}
 = \sum_{j=k-1}^{d}c_ka^{j-(k-1)}.
\]
On a donc $r_0 = \sum_{j=0}^{d}c_ja^{j} = P(a)$, puis $k$ passe à $-1$, et au test suivant la boucle s'arrête.  L'algorithme retourne alors la bonne valeur.

\end{document}
