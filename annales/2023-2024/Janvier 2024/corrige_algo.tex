
\documentclass[10pt,a4paper]{article}
\usepackage[utf8]{inputenc}
\usepackage[T1]{fontenc}
\usepackage[french]{babel}
\usepackage[top=3cm, bottom=2cm, left=2cm, right=2cm]{geometry}
\usepackage{stmaryrd}
\usepackage{amsmath}
\usepackage{amsfonts}
\usepackage{amssymb}
\usepackage{mathrsfs}
\usepackage{amsthm}
\usepackage{layout}
\usepackage{fancyhdr}

\newtheorem*{thm}{Théorème}
\newtheorem{ex}{Exercice}
\newtheorem*{nota}{Notation}
\newtheorem*{rem}{Remarque}
\newtheorem*{rem2}{Remarques}
\newtheorem{de2}{Définition}
\newtheorem{pro2}[de2]{Propriété}
\newtheorem{thm2}[de2]{Théorème}

\setlength{\parindent}{0cm}
\setlength{\parskip}{1ex plus 0.5ex minus 0.2ex}
\newcommand{\hsp}{\hspace{20pt}}
\newcommand{\HRule}{\rule{\linewidth}{0.5mm}}

\usepackage{comment}
\usepackage{xr-hyper}

\title{}

\date{}



\makeatletter%pour faire référence aux exercices de examen.tex
\newcommand*{\addFileDependency}[1]{% argument=file name and extension
  \typeout{(#1)}
  \@addtofilelist{#1}
  \IfFileExists{#1}{}{\typeout{No file #1.}}
}
\makeatother

\newcommand*{\myexternaldocument}[1]{%
    \externaldocument{#1}%
    \addFileDependency{#1.tex}%
    \addFileDependency{#1.aux}%
}
\myexternaldocument{examen_algo}%pour faire référence aux exercices de enonce.tex


\begin{document}




\pagestyle{fancy}

\fancyhead{}
 \fancyfoot{}

 \lhead{ 2023/2024 \\  L3 Mathématiques
}
\chead{\textbf{Algorithmes pour l'enseignement}\\} 
 \rhead{ Université de Lorraine \\  }

\newcommand{\lb}{\llbracket}
\newcommand{\rb}{\rrbracket}
\newcommand{\N}{\mathbb{N}}
\newcommand{\R}{\mathbb{R}}
\newcommand{\Z}{\mathbb{Z}}




\newcommand{\md}[3]{#1\ \equiv \ #2 \! \! \! \! \! \pmod {#3} }
\newcommand{\nmd}[3]{#1 \not \equiv #2 \! \! \! \! \!  \pmod {#3} }
\newcommand{\mda}[3]{#1 \equiv #2 \! \!  \pmod {#3} }
\newcommand{\nmda}[3]{#1 \not \equiv #2 \! \! \pmod {#3} }
\newcommand{\mo}[2]{#1 \! \! \! \! \! \pmod #2 }
\newcommand{\moa}[2]{#1 \! \!  \pmod {#2} }

\thispagestyle{fancy}

\begin{center}
%    \HRule \\[0.6cm]
    { \huge \bfseries
Corrigé
     \\ [0cm] }
    \HRule \\[0.5cm]
\end{center}






\textbf{Exercice~\ref{prelim}}   1)

\begin{tabular}{ll}
\textbf{Algorithme} &Valeur\_absolue($x$) \\
& si $x<0$:\\
&\ \ \ {\ \rm |}$x\leftarrow -x$\\
& Renvoyer $x$.
\end{tabular}

2)

\begin{tabular}{ll}
\textbf{Algorithme} &Factorielle($n$) \\
&$a\leftarrow 1$\\
&  pour $i$ de $1$ à $n$:\\
&\ \ \ {\ \rm |} $a\leftarrow ai$\\
& Renvoyer $a$.
\end{tabular}


3)

\begin{tabular}{ll}
\textbf{Algorithme} &Puissance($n,x$) \\
&$a\leftarrow 1$\\
&  pour $i$ de $1$ à $n$:\\
&\ \ \ {\ \rm |} $a\leftarrow ax$\\
& Renvoyer $a$.
\end{tabular}

\begin{tabular}{ll}
\textbf{Algorithme} &Puissance\_rec($n,x$) \\
&si $n=0$:\\
& \ \ \ {\ \rm |} Renvoyer $1$:\\
& sinon :\\
& \ \ \ {\ \rm |} Renvoyer $x$Puissance\_rec($n-1,x$)
\end{tabular}


4)

 \begin{tabular}{ll}
\textbf{Algorithme} &Somme\_serie($n,x$) \\
& $a\leftarrow 0$ \\
& pour $i$ de $0$ à $n$:\\
& \ \ \ {\ \rm |}  $a\leftarrow a+$Puissance$(i,x)$/Factorielle($i$)\\
& Renvoyer $a$.
\end{tabular}

\begin{tabular}{ll}
\textbf{Algorithme} &Somme\_serie2($n,x$) \\
& $a\leftarrow 0$ \\
& $y\leftarrow 1$\\
& pour $i$ de $1$ à $n$:\\
& \ \ \ {\ \rm |} $y\leftarrow y. \frac{x}{i}$\\
& \ \ \ {\ \rm |}  $a\leftarrow a+y$\\
& Renvoyer $a$.
\end{tabular}


\textbf{Exercice~\ref{exSuite_recurrente}}

1) 

 \begin{tabular}{ll}
\textbf{suite\_u}($n$):\\
& $a\leftarrow 1$ \ \ \ \ \# $a$ correspond à $u_n$ \\ 
& $b\leftarrow 2$ \ \ \ \ \# $b$ correspond à $u_{n+1}$ \\
& Pour $i$ allant  de $1$ à $n-1$: \\
&\ \ \ {\ \rm |} $a,b\leftarrow b,3a+2b$\\
& Si $n=0$:\\
&\ \ \ {\ \rm |}Renvoyer $a$\\
& Sinon: \\
&\ \ \ {\ \rm |}Renvoyer $b$.
\end{tabular}

2) 

 \begin{tabular}{ll}
\textbf{suite\_u\_rec}($n$):\\

& Si $n<=1$:\\
&\ \ \ {\ \rm |}Renvoyer $n+1$\\
& Sinon: \\
&\ \ \ {\ \rm |}Renvoyer $3*suite\_u\_rec(n-2)+2*suite\_u\_rec(n-1)$.
\end{tabular}

3) \begin{tabular}{ll}
\textbf{suite\_v}($n$):\\

& Si $n<=1$:\\
&\ \ \ {\ \rm |}Renvoyer $1+2n$\\
& sinon si $n\%2=0$:\\
&\ \ \ {\ \rm |}Renvoyer (suite\_v$(n/2))^2+5$\\
& sinon:\\
&\ \ \ {\ \rm |}Renvoyer (suite\_v$((n-1)/2))$.(suite\_v$((n-1)/2+1))+7$.\\
\end{tabular}

\textbf{Exercice~\ref{exTri_selection} : }cf cours.


\textbf{Exercice~\ref{eqPell-fermat}}

Dans le programme suivant, $\mathrm{Ent}$ désigne la fonction partie entière inférieure. 

\begin{tabular}{ll}
\textbf{solutions}():\\
&$L\leftarrow [\ ]$\\
& Pour $y$ de $1$ à $100$\\
& \ \ \ {\ \rm |} Si $\mathrm{Ent}(\sqrt{1+2y^2})=\sqrt{1+2y^2}$:\\
& \ \ \ \ \ \ {\ \rm |} $L\leftarrow L+[(\sqrt{1+2y^2},y)]$\\
& Renvoyer $L$.\\
\end{tabular}


\textbf{Exercice~\ref{exSuite_e}}

La suite $(u_n)$ est strictement croissante car $u_{n+1}-u_n>0$ pour tout $n\in \N^*$. Soit $n\in \N^*$. Alors $v_{n+1}-v_n=\frac{1}{(n+1)^2 n!}-\frac{1}{n.n!}=\frac{1}{n!}(\frac{1}{(n+1)^2}-\frac{1}{n})<0$. La suite $(v_n)$ est donc strictement décroissante. Comme $v-u$ tend vers $0$, les suites $u$ et $v$ sont adjacentes.


Pour tout $n\in \N$, on a $u_n< e <v_n=u_n+\frac{1}{n.n!}$, donc $0<e-u_n<\frac{1}{n.n!}$. On peut donc utiliser le programme suivant :


\begin{tabular}{ll}
\textbf{approximation\_e}($\epsilon$):\\
& $s\leftarrow 2$\\
& $P\leftarrow 1$\\
& $n\leftarrow 1$\\
& Tant que $1/(P*n)>\epsilon$:\\
& \ \ \ {\ \rm |}$n\leftarrow n+1$\\
& \ \ \ {\ \rm |}$P\leftarrow P*n$\\
& \ \ \ {\ \rm |}$s\leftarrow s+1/P$\\
& Renvoyer $s$. 
\end{tabular}






\textbf{Exercice~\ref{Capes}}

1) Soit $f:\R\rightarrow \R$ définie par $f(x)=\frac{1}{2}x+1$, pour $x\in  \R$. Soit $(u_n)\in \R^\N$ la suite définie par $u_0=-3$, $u_{n+1}=f(u_n)$, pour $n\in \N$. Alors comme $f$ est croissante, $(u_n)$ est monotone. De plus, $u_1=-\frac{1}{2}$, donc $(u_n)$ est croissante et admet une limite $\ell\in \overline{\R}$. Si $\ell$ est finie, alors $u_{n+1}=f(u_n)\rightarrow f(\ell)$ (car $f$ est continue) et $u_{n+1}\rightarrow \ell$, donc $\ell=f(\ell)$, donc $\ell=2$. Ainsi, soit $(u_n)$ est majorée par $2$, soit $u_n\rightarrow +\infty$. Soit $n\in \N$ tel que $u_n \leq 2$. Alors $u_{n+1}\leq \frac{1}{2}.2+1=2$, donc $u_{n+1}\leq 2$. Par récurrence, on en déduit que $(u_n)$ est majorée par $2$, et donc qu'elle converge en croissant vers $2$. En particulier, l'algorithme termine, puisque $A=u_N$, après chaque passage dans la boucle. 

2)
\begin{tabular}{ll}
 \textbf{Algorithme} &Lancers\\
& $A\leftarrow $AléaEntre($1,6$) \\
& $I\leftarrow 1$\\
& Tant que $A\neq 6$\\
& \ \ \ {\ \rm |} $I\leftarrow I+1$\\
&  \ \ \ {\ \rm |}$A\leftarrow $AléaEntre($1,6$)\\
& Renvoyer $I$.
\end{tabular}

À la fin des deux étapes de la boucle  «~tant que~», $I$ vaut le nombre de lancers effectués. 






\end{document}


