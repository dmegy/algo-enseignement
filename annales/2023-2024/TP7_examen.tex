\documentclass[11pt,a4paper]{article}
\usepackage[utf8]{inputenc}
\usepackage[T1]{fontenc}
\usepackage[top=3cm, bottom=2cm, left=2cm, right=2cm]{geometry}
\usepackage{stmaryrd}
\usepackage{amsmath}
\usepackage{amsfonts}
\usepackage{amssymb}
\usepackage{mathrsfs}
\usepackage{amsthm}
\usepackage[french]{babel}
\usepackage{layout}
\usepackage{fancyhdr}
\usepackage{stmaryrd}
\usepackage{graphicx}
\usepackage{float}
\newtheorem*{thm}{Théorème}
\newtheorem{ex}{Exercice}
\newtheorem*{nota}{Notation}
\newtheorem*{rem}{Remarque}
\newtheorem*{rem2}{Remarques}
\newtheorem{de2}{Définition}
\newtheorem{pro2}[de2]{Propriété}
\newtheorem{thm2}[de2]{Théorème}
\setlength{\parindent}{0cm}
\setlength{\parskip}{1ex plus 0.5ex minus 0.2ex}
\newcommand{\hsp}{\hspace{20pt}}
\newcommand{\HRule}{\rule{\linewidth}{0.5mm}}
\title{}

\date{}
\begin{document}


\pagestyle{fancy}

\fancyhead{}
 \fancyfoot{}

 \lhead{ 2023/2024 \\  L3 Mathématiques
}
\chead{\textbf{Algorithmes pour l'enseignement}\\} 
 \rhead{  Université de Lorraine \\ }

\newcommand{\lb}{\llbracket}
\newcommand{\rb}{\rrbracket}

\newcommand{\md}[3]{#1 \equiv #2 \! \! \! \! \! \pmod {#3} }
\newcommand{\nmd}[3]{#1 \not \equiv #2 \! \! \! \! \!  \pmod #3 }
\newcommand{\mda}[3]{#1 \equiv #2 \! \!  \pmod #3 }
\newcommand{\nmda}[3]{#1 \not \equiv #2 \! \! \pmod #3 }
\newcommand{\mo}[2]{#1 \! \! \! \! \! \pmod #2 }
\newcommand{\N}{\mathbb{N}}
\newcommand{\Z}{\mathbb{Z}}
\newcommand{\R}{\mathbb{R}}


\thispagestyle{fancy}

\begin{center}
%    \HRule \\[0.6cm]
    { \huge \bfseries
    TP n$^{\boldsymbol{\circ}}$7 : Évaluation
     \\ [0cm] }
    \HRule \\[0.5cm]
\end{center}


\begin{ex}
\begin{enumerate}
\item Écrire un programme qui prend en entrée $n\in \N$ et qui renvoie $n!$. Écrire une version itérative et une version récursive.

\item Écrire un programme qui prend en entrée $x\in \R$, $n\in \N$ et qui renvoie $f_n(x)=\sum_{k=0}^n \frac{x^k}{k!}$. 

\item Représenter sur un même graphe $f_5, f_{10}$ et $\exp$ entre $-10$ et $10$. Mettre une légende (qui indiquera à quoi correspondent chaque courbe, par exemple $y=\exp(x)$), représenter $\exp$ en rouge, $f_5$ en vert et $f_{10}$ en bleu, et représenter $\exp$ avec un trait plus épais que $f_5$ et $f_{10}$.

\end{enumerate}
\end{ex}


\begin{ex}
\begin{enumerate}


\item Faire la liste des $50$ premiers  nombres premiers.

\item Un nombre premier de Sophie Germain est un nombre premier $p$ tel que $2p+1$ est premier. Faire la liste des $50$ premiers nombres premiers de Sophie Germain.
\end{enumerate}

\end{ex}

\begin{ex}

Soit la suite de Fibonacci, définie par~:

\begin{displaymath}
  \left\{
    \begin{array}[l]{rcl}
      u_0 &=& 0\\
      u_1 &=& 1\\
      u_{n} &=& u_{n-1} + u_{n-2},  \quad \forall n\geq 2.
    \end{array}\right.
\end{displaymath}

\begin{enumerate}



\item Écrire un programme \texttt{Fibo\_modulaire}(m,n) qui prend en entrée deux entiers positifs $m,n$ et qui renvoie la liste des $F_i\%m$, pour $i\in \llbracket 0,n\rrbracket$. 

\item Écrire un programme \texttt{triplet}(L) qui prend en entrée une liste $L$ de nombres $L=[l_0,\ldots,l_n]$ et qui renvoie la liste des triplets de la forme $(l_i,l_{i+1},l_{i+2})$,  tels que $i\in \llbracket 0,n-2\rrbracket$ et $l_{i+1}<l_{i}<l_{i+2}$.

\item Écrire un programme \texttt{alea}($n$), qui prend en entrée un entier $n$ et qui renvoie une liste de $n$ éléments aléatoires entre $1$ et $10^5$ (on pourra utiliser la commande \texttt{random.randint}).

\item Si on prend un triplet de réels « au hasard » $x_0,x_1,x_2$, l'inégalité $x_1<x_0<x_2$ doit se produire une fois sur $6$. Combien de triplets obtenez vous avec la commande \texttt{triplets}(\texttt{alea}($10^5$)) ?

\item Combien de triplets obtenez-vous avec la commande  \texttt{triplets}(\texttt{fibo\_modulaire}($1000,10^5$)) ?
\end{enumerate}
\end{ex}


\begin{ex}
 Écrire un programme qui prend en entrée  un entier $m$ et une liste   $M$ d'entiers de $\llbracket 0,m-1\rrbracket$ et  qui renvoie la liste $N=[n_0,,\ldots,n_{m-1}]$ telle que pour tout $i\in \llbracket 0,m-1\rrbracket$, $n_i=|\{j\in \llbracket 0,m-1\rrbracket \mid M[j]=i\}|$. 

\end{ex}

\begin{ex}
\begin{enumerate}
\item Écrire un programme qui prend en entrée deux listes triées $L_1$ et $L_2$ et qui renvoie la liste triée contenant la réunion des deux listes.
\item En déduire un programme qui prend une liste en entrée et qui la trie à l'aide d'un tri fusion.
\end{enumerate}
\end{ex}



\begin{ex}(méthode des rectangles)

Soient $a,b\in \R$ et $f:[a,b]\rightarrow \R$ une fonction de classe $\mathcal{C}^1$. Pour $n\in \N^*$, on pose $S_n(f)=\frac{1}{n}\sum_{i=0}^{n-1} f(a+i(b-a)/n)$. On rappelle que $S_n(f)\underset{n\to \infty}{\rightarrow} \int_a^b f$ (et que $S_n(f)-\int_a^b f=O(\frac{1}{n})$).  On suppose que l'on a défini une fonction $f$ sur Python (à l'aide de la fonction def f(x): ... return ...).

\begin{enumerate}
\item Écrire un programme qui prend en entrée  $a,b\in \R$ et $n\in \N^*$ et qui renvoie $S_n(f)$. 

\item Calculer $S_n(f)$ pour différentes valeurs de $n$, pour $f=\exp$, $a=0$, $b=1$. Comparer avec le résultat   que l'on obtient avec la fonction \texttt{quad}(f,a,b) (après avoir importé "quad" de scipy.integrate).

\item Écrire un programme prenant en entrée $a,b,n$ et qui renvoie une figure représentant le graphe de $f$ sur $[a,b]$ ainsi que les rectangles correspondant à $S_n(f)$. Les rectangles devront être tous d'une même couleur et en pointillés.
\end{enumerate}

\end{ex}


\end{document}
