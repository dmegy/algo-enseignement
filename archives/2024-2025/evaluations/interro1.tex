
\documentclass[11pt,a4paper]{article}
\usepackage[utf8]{inputenc}
\usepackage[T1]{fontenc}
\usepackage[french]{babel}
\usepackage[top=3cm, bottom=2cm, left=2cm, right=2cm]{geometry}
\usepackage{stmaryrd}
\usepackage{amsmath}
\usepackage{amsfonts}
\usepackage{amssymb}
\usepackage{mathrsfs}
\usepackage{amsthm}
\usepackage{layout}
\usepackage{fancyhdr}

\newtheorem*{thm}{Théorème}
\newtheorem{ex}{Exercice}
\newtheorem*{nota}{Notation}
\newtheorem*{rem}{Remarque}
\newtheorem*{rem2}{Remarques}
\newtheorem{de2}{Définition}
\newtheorem{pro2}[de2]{Propriété}
\newtheorem{thm2}[de2]{Théorème}

\setlength{\parindent}{0cm}
\setlength{\parskip}{1ex plus 0.5ex minus 0.2ex}
\newcommand{\hsp}{\hspace{20pt}}
\newcommand{\HRule}{\rule{\linewidth}{0.5mm}}

\usepackage{comment}

\usepackage{listings}
\usepackage{xcolor}

\definecolor{codeblue}{rgb}{0,0,0.9}
\definecolor{codegreen}{rgb}{0,0.6,0}
\definecolor{codegray}{rgb}{0.5,0.5,0.5}
\definecolor{codepurple}{rgb}{0.58,0,0.82}
\definecolor{backcolour}{rgb}{0.95,0.95,0.92}

\lstdefinestyle{mystyle}{
    backgroundcolor=\color{backcolour},   
    commentstyle=\color{codegreen},
    keywordstyle=\color{codeblue},
    %numberstyle=\tiny\color{codegray},
    stringstyle=\color{codepurple},
    basicstyle=\ttfamily\footnotesize,
    breakatwhitespace=false,         
    breaklines=true,                 
    captionpos=b,                    
    keepspaces=true,                 
    %numbers=left,                    
    %numbersep=5pt,                  
    showspaces=false,                
    showstringspaces=false,
    showtabs=false,                  
    tabsize=2
}

\lstset{style=mystyle}

\title{}

\date{}
\begin{document}


\pagestyle{fancy}

\fancyhead{}
 \fancyfoot{}

 \lhead{ 2024/2025 \\  L3 Mathématiques
}
\chead{\textbf{ Algorithmes pour l'enseignement}\\} 
 \rhead{ Université de Lorraine \\  }

\newcommand{\lb}{\llbracket}
\newcommand{\rb}{\rrbracket}
\newcommand{\N}{\mathbb{N}}
\newcommand{\Z}{\mathbb{Z}}
\newcommand{\R}{\mathbb{R}}




\newcommand{\md}[3]{#1\ \equiv \ #2 \! \! \! \! \! \pmod {#3} }
\newcommand{\nmd}[3]{#1 \not \equiv #2 \! \! \! \! \!  \pmod {#3} }
\newcommand{\mda}[3]{#1 \equiv #2 \! \!  \pmod {#3} }
\newcommand{\nmda}[3]{#1 \not \equiv #2 \! \! \pmod {#3} }
\newcommand{\mo}[2]{#1 \! \! \! \! \! \pmod #2 }
\newcommand{\moa}[2]{#1 \! \!  \pmod {#2} }

\thispagestyle{fancy}

\begin{center}
%    \HRule \\[0.6cm]
    { \huge \bfseries
Interrogation du 30 septembre 2024
     \\ [0cm] }
    \HRule \\[0.5cm]
\end{center}

\begin{center}
\textbf{durée : 1 heure}
\end{center}



Pour chacun des algorithmes, on justifiera avec soin : \begin{itemize}
\item[•] que l'algorithme termine (si une boucle «~tant que~» est utilisée).

\item[•] que l'algorithme renvoie bien le résultat demandé.
\end{itemize}

Une réponse non justifiée sera notée sur les trois quarts des points.

On peut utiliser les opérateurs \verb+//+ et \verb+%+ sur les entiers. On évitera d'utiliser des fonctionnalités trop avancées de python, surtout celles non vues en TD : on rédigera en pseudo-code avec des techniques élémentaires. En cas de doute, poser simplement la question.




\bigskip

Pour les boucles «~pour~», on adoptera les conventions suivantes. Si $a,b\in \Z$, «~pour $i$ allant de $a$ à $b$~» signifie «~pour $i$ parcourant en croissant l'intervalle $\llbracket a,b\rrbracket$~». Lorsque $b<a$, cet intervalle est vide, donc aucune des instructions dans la boucle «~pour~» n'est effectuée (ce sera par exemple le cas si on écrit «~pour $i$ allant de $1$ à $n$~», avec $n=0$). On n'utilisera pas d'autre type de boucles «~pour~» (décroissantes, saut d'indice etc) : dans ces situations, on utilisera l'instruction «~tant que~».

%On pourra utiliser l'instruction «~pour $i$ allant en décroissant de $b$ à $a$~», qui signifie (si $b\geq a$) «~pour $i$ prenant successivement les valeurs $b$, $b-1$,$\ldots$, $a$~». Lorsque $a>b$, aucune instruction dans la boucle «~pour~» n'est alors effectuée.





\begin{ex}
Écrire un algorithme itératif, puis un autre récursif, recevant un entier $n$ naturel en entrée, et retournant $n!$ en sortie.
\end{ex}

\begin{ex}
Écrire un algorithme itératif, puis un autre récursif, recevant une liste non vide de nombres en entrée, et retournant la valeur minimale dans cette liste.
\end{ex}


\begin{ex}\label{base10}
Écrire un algorithme itératif, puis un autre récursif, recevant un entier naturel et retournant la liste de ses chiffres en base $10$ (dans l'ordre usuel : le chiffre des unités en dernier).
\end{ex}

Dans la suite, on utilise des chaînes de caractères. On les manipule comme des listes, en particulier la concaténation sera notée \og $+$\fg, et la longueur s'obtient avec une fonction  \verb+longueur+ que l'on peut librement utiliser. Les différentes lettres d'une chaîne de caractères sont accessibles par leur indice comme une liste : $c[0]$, $c[1]$ etc.
Une chaîne de caractères \textbf{peut être vide}, auquel cas elle est de longueur zéro.

\begin{ex}\label{est_un_carre}
On dit qu'une chaîne de caractères $s$ est un carré s'il existe une chaîne de caractères $a$ telle que $s$ soit égale à $a$ concaténée avec elle-même. Par exemple, \verb+'papa'+ est un carré, mais pas \verb+'maman'+. 
\begin{enumerate}
\item Écrire une fonction \verb+est_un_carre(c)+ qui prend en entrée une chaîne de caractères et qui retourne \verb+True+ si c'est un carré et \verb+False+ sinon.
\item Si la chaîne en entrée est de longueur $n$, majorer le nombre de comparaisons entre lettres effectuées par la fonction dans le pire des cas.
\end{enumerate}
\end{ex}

\begin{ex}\label{contient_un_carre}
On dit qu'une chaîne de caractères $s$ contient un carré si c'est la concaténation de trois chaînes de caractères $a$, $b$ et $c$ (dans cet ordre) et que $b$ est un carré \textbf{non vide}.
Par exemple, \verb+'maman'+ contient un carré (prendre $a$ la chaîne vide, $b$ la chaîne \verb+'mama'+ qui est bien un carré et $c$ la chaîne \verb+'n'+). 
\begin{enumerate}
\item Écrire une fonction \verb+contient_un_carre(s)+ qui prend en entrée une chaîne de caractères et qui retourne \verb+True+ si elle contient un carré et \verb+False+ sinon. On pourra utiliser la fonction \verb+est_un_carre+ de l'exercice précédent.
\item Si la chaîne en entrée est de longueur $n$, majorer le nombre de comparaisons entre lettres effectuées par la fonction dans le pire des cas.
\end{enumerate}
\end{ex}



\begin{ex}\label{pythagore}
Un triplet pythagoricien ordonné est un triplet d'entiers naturels non nuls $(a,b,c)$ vérifiant $a\leq b \leq c$ et $a^2+b^2 = c^2$.

Écrire une fonction \verb+triplets_pyathgoriciens(N)+ prenant un entier $N$ naturel en entrée et retournant la liste de tous les triplets pythagoriciens ordonnés $(a,b,c)$ avec $c\leq N$. Chaque triplet pythagoricien sera retourné sous forme de liste, donc la fonction doit retourner une liste de listes. Par exemple, \verb+triplets_pyathgoriciens(2)+ doit retourner $[]$, \verb+triplets_pyathgoriciens(6)+ doit retourner $[[3,4,5]]$ (liste contenant un unique triplet pythagoricien ordonné), et 
\verb+triplets_pyathgoriciens(13)+ doit retourner une liste contenant trois éléments, ces éléments devant être $[3,4,5]$, $[6,8,10]$ et $[5,12,13]$ dans un ordre que l'on n'impose pas.
\end{ex}

\end{document}