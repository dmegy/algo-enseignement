  \documentclass[12pt]{article}
\usepackage{a4wide}
\usepackage[utf8]{inputenc}
\usepackage[french]{babel}
\usepackage[T1]{fontenc}
\usepackage{stmaryrd}
\usepackage{amsmath}
\usepackage{amsfonts}
\usepackage{amssymb}
\usepackage{mathrsfs}
\usepackage{amsthm}
\usepackage{ulem}
\usepackage{listings}
\usepackage{listings}
\usepackage{xcolor}

\definecolor{codeblue}{rgb}{0,0,0.9}
\definecolor{codegreen}{rgb}{0,0.6,0}
\definecolor{codegray}{rgb}{0.5,0.5,0.5}
\definecolor{codepurple}{rgb}{0.58,0,0.82}
\definecolor{backcolour}{rgb}{0.95,0.95,0.92}

\lstdefinestyle{mystyle}{
    backgroundcolor=\color{backcolour},   
    commentstyle=\color{codegreen},
    keywordstyle=\color{codeblue},
    %numberstyle=\tiny\color{codegray},
    stringstyle=\color{codepurple},
    basicstyle=\ttfamily\footnotesize,
    breakatwhitespace=false,         
    breaklines=true,                 
    captionpos=b,                    
    keepspaces=true,                 
    %numbers=left,                    
    %numbersep=5pt,                  
    showspaces=false,                
    showstringspaces=false,
    showtabs=false,                  
    tabsize=2
}

\lstset{style=mystyle}
\usepackage[dvipsnames]{xcolor}
\usepackage[colorlinks=true]{hyperref} %pour faire des liens hypertextes

\bibliographystyle{alpha}%pour que les références soient une partie du nom+l'année
\long\def\rge#1{{\color{red}#1}}
\long\def\bl#1{{\color{blue}#1}}
\long\def\vrt#1{{\color{ForestGreen}#1}}


\setcounter{secnumdepth}{3}
\setcounter{tocdepth}{2}
\usepackage{graphicx}












\newtheorem*{att}{Attention !!!}
\newtheorem*{de}{Définition}
\newtheorem*{qst}{Question}
\newtheorem*{rep}{Réponse}
\newtheorem*{reps}{Réponses}
\newtheorem{thm}{Théorème}[section]
\newtheorem{theoreme}[thm]{Théorème}
\newtheorem{algo}{Algorithme}
\newtheorem*{rap}{Rappels}
\newtheorem{sch}[thm]{Schéma}
\newtheorem*{ex}{Exemple}
\newtheorem*{exo}{Exercice}
\newtheorem*{exs}{Exemples}
\newtheorem*{rem}{Remarque}
\newtheorem*{rems}{Remarques}
\newtheorem{remarque}[thm]{Remarque}
\newtheorem{remarques}[thm]{Remarques}
\newtheorem*{but}{But}
\newtheorem{Def}[thm]{Définition}
\newtheorem{definition}[thm]{Définition}
\newtheorem{Def/prop}[thm]{Définition/Proposition}
\newtheorem{Lem}[thm]{Lemme}
\newtheorem{lemme}[thm]{Lemme}
\newtheorem{exemple}[thm]{Exemple}
\newtheorem{exemples}[thm]{Exemples}
\newtheorem{Prop}[thm]{Proposition}
\newtheorem{proposition}[thm]{Proposition}
\newtheorem*{prop}{Proposition}
\newtheorem*{prin}{Principe}
\newtheorem*{nota}{Notation}
\newtheorem*{notas}{Notations}

\newcommand{\blue}[1]{{\color{blue}#1}}
\newcommand{\red}[1]{{\color{red}#1}}
\newcommand{\green}[1]{{\color{green}#1}}
\newcommand{\yellow}[1]{{\color{yellow}#1}}
\newcommand{\magenta}[1]{{\color{magenta}#1}}


\newtheorem{Cor}[thm]{Corollaire}
\newtheorem{corollaire}[thm]{Corollaire}
\usepackage{comment}



\title{Ordres bien fondés et application à la terminaison d'algorithmes}
\author{Damien Mégy}


\newcommand{\N}{\mathbb{N}}
\newcommand{\Q}{\mathbb{Q}}
\newcommand{\R}{\mathbb{R}}
\newcommand{\Z}{\mathbb{Z}}
\newcommand{\PCC}{\mathscr{P}}
\newcommand{\cont}{\mathrm{cont}}
\newcommand{\cd}{\mathrm{cd}}



\begin{document}


\maketitle

\tableofcontents

\section{Motivation}

Dans ce chapitre, on présente une notion théorique, celle d'\emph{ordre bien fondé}, ainsi qu'un principe de récurrence généralisé, valable sur les ensembles bien fondés et non plus seulement $\N$.

Ensuite, on applique ceci à la rédaction de preuves de terminaison de certains algorithmes.

\section{Prérequis}

Chapitre de relations d'ordre de première année. Relations d'ordre, ordre total, plus petit élément, majorants et minorants, applications croissantes.

Sources possibles pour réviser : 
\begin{enumerate}
\item Tout-en-un L1 Ramis-Warusfel à la BU.
\item Chapitre sur les relations d'équivalence et d'ordre sur Bibmath : \url{https://www.bibmath.net/ressources/index.php?action=affiche&quoi=bde/logique/relations&type=fexo}. Les derniers exercices correspondent au début de ce chapitre de compléments. (Mais les définitions prises ne sont pas forcément les mêmes.)
\item Mansuy, Tout-en-un MPSI, livre \og anciens programmes\fg{} épuisé et  mis à disposition gratuitement par l'auteur : \url{http://www.rogermansuy.fr/pdf/MPSI2019.pdf}
\item Document d'accompagnement de l'ancienne UE de L1 couvrant ce thème à Nancy : \url{https://github.com/dmegy/decouverteDesMaths/blob/master/cours.pdf}. Ce poly un peu expérimental est un bon moyen de réviser les fondamentaux de L1 quelques années après. Certains exercices sont assez difficiles. Tous ces exercices ont été ajoutés à la base exo7, qui en contenait déjà beaucoup d'autres plus classiques.
\item Exo 7 : \url{http://exo7.emath.fr/bin/extract5.php}, catégorie L1 Algèbre /100.05. 
\end{enumerate}

\section{Introduction et rappels}

\begin{definition}
Soit $\leq_E$ un ordre sur un ensemble $E$.
L'\textbf{ordre strict} associé à $\leq$ est la relation binaire $<_E$ sur $E$ définie par 
\[ x <_E y \iff x\leq_E y \text{ et } x\neq y.\]
 \end{definition}

\paragraph{Exemples  et contre-exemples fondamentaux de relations d'ordre}
\begin{enumerate}
\item L'ensemble $\N$ muni de la relation d'ordre usuelle $\leq$ est un ensemble ordonné. 
\item L'ensemble $\N$ muni de la relation de divisibilité est un ensemble ordonné.
\item L'ensemble $\Z$ muni de la relation de divisibilité n'est \textbf{pas} un ensemble ordonné.
\item Si $E$ est un ensemble, l'ensemble $\mathcal P(E)$ des parties de $E$ est naturellement  ordonné par la relation $\subseteq$.
\item Si $(E,\leq_E)$ et $(F,\leq_F)$ sont des ensembles ordonnés, l'ensemble produit $E\times F$ peut être muni:
\begin{enumerate}
\item de l'\textbf{ordre produit}, définie par :
\[ (e,f) \leq_{prod} (e',f') \iff e\leq_E e' \text{ et }f\leq_F f';\]
\item de l'\textbf{ordre lexicographique}, défini par 
\[ (e,f) \leq_{lex} (e',f') \iff (e <_E e' \text{ ou } (e=e' \text{ et }f\leq_F f') ).\]
\end{enumerate}
\end{enumerate}

 
 
 \begin{definition}
 Un ordre $\leq$ sur $E$ est dit \emph{total} si tous les éléments sont comparables : 
 \[ \forall (x,y)\in E^2, x\leq y \text{ ou } y\leq x.\]
 \end{definition}
 
\begin{exo}
\begin{enumerate}
\item Montrer que l'ordre produit sur $\N^2$ n'est pas total.
\item Montrer que la divisibilité sur $\N$ n'est pas un ordre total.
\item Montrer que l'ordre lexicographique sur $\N^2$ est total.
%\item Montrer que si $E$ et $F$ sont totalement ordonnés, l'ordre lexicographique sur le produit $E\times F$ est un ordre total.
\end{enumerate}
\end{exo}


Une suite dans un ensemble $E$ est un élément de $E^\N$, c'est-à-dire une application de $\N$ dans $E$.
On note en général les suites comme des familles indexées par $\N$, par exemple comme ceci :  $u = (u_n)_{n\in\N}$.


Si l'ensemble $E$ est muni d'une relation d'ordre $\leq_E$, une suite est (strictement) (dé)croissante si l'application en question est (strictement) (dé)croissante, vue comme application entre les ensembles ordonnés $(\N,\leq)$ (ordre usuel) et $(E,\leq_E)$.



\section{Ordres bien fondés}
 
\begin{definition}
Soit $(E,\leq)$ un ensemble ordonné, $X\subseteq E$ une partie de $E$ et $a\in X$.
On dit que : 
\begin{enumerate}
\item $a$ est un plus petit élément de $X$ si $\forall x \in X, a\leq x$.
\item $a$ est un élément minimal de $X$ s'il n'existe pas de $y\in X$ avec $x < a$.
\end{enumerate}
(On définit de même la notion de plus grand élément et d'élément maximal.)
\end{definition}

\paragraph{Mise en garde} 
Dans beaucoup de références, on voit \og élément minimum\fg{} au lieu de \og plus petit élément\fg. Attention alors à ne pas confondre élément \emph{minimum} et \emph{minimal}.

\begin{exemples}
\begin{enumerate}
\item Dans $\N\setminus\{0,1\}$ muni de la divisibilité, $5$ est minimal mais n'est pas un plus petit élément.
\item Un plus petit élément est unique, un élément minimal pas forcément.
\item Dans un ensemble totalement ordonné, un élément minimal est un (le) plus petit élément.
\end{enumerate}
\end{exemples}
  
\begin{definition}
Soit $(E,\leq)$ un ensemble ordonné. On dit que : 
\begin{enumerate}
\item $(E,\leq)$ est \emph{bien ordonné} si toute partie non vide admet un plus petit élément.
\item $(E,\leq)$ est \emph{bien fondé} si toute partie non vide admet un élément minimal.
\end{enumerate}
\end{definition}
 
\begin{exemples}
\begin{enumerate}
\item Un bon ordre est toujours bien fondé.
\item L'ordre usuel $\leq$ sur $\Z$ n'est pas un bon ordre, ni bien fondé : les parties $\Z$, ou $2\Z$, ou $\Z_-$, n'ont pas de plus petit élément, ni d'élément minimal.
\item L'ordre usuel sur $\N$ est un bon ordre.
\item La divisibilité sur $\N$ est bien fondée, mais n'est pas un bon ordre : la partie $\{2,3\}$ n'a pas de plus petit élément pour la divisibilité.
\item l'ensemble $\mathcal P(\N)$ avec l'ordre usuel induit par l'inclusion de parties n'est pas bien fondé : la partie formée par les parties $[n,\infty[$ n'a pas d'élément minimal.
\end{enumerate}
\end{exemples}
 
\begin{exo}
Montrer qu'un ordre est bon ssi il est bien fondé et total.
\end{exo}


\begin{thm}
Un ensemble ordonné $E$ est bien fondé si et seulement s'il n'existe pas de suite strictement décroissante dans $E$.
\end{thm}


\begin{proof}
\begin{enumerate}
\item Supposons qu'il existe une suite $(u_n)_{n\in\N}$ d'éléments de $E$, strictement décroissante. Alors, l'image de la suite $X = \{u_n, n\in\N\}$ est une partie non vide qui n'a pas d'élément minimal. En effet, si $x\in X$, alors par définition de $X$ il existe $n\in \N$ tel que $x=u_n$. Définissons alors $y=u_{n+1}$. On a $y \in X$ et  $y<x$ car la suite est strictement décroissante.
\item Réciproquement, s'il existe une partie non vide $X\subseteq E$ sans élément minimal, on peut construire par récurrence une suite $(u_n)_{n\in\N}$ strictement décroissante en prenant $u_0\in X$ puisque $X$ est non vide, puis comme $X$ n'a pas d'élément minimal, $u_0$ n'est pas minimal donc il existe $u_1 \in X$ avec $u_1 < u_0$, mais $u_1$ n'est pas non plus minimal etc.
\end{enumerate}
\end{proof}

\begin{exo}
Un ensemble est bien fondé ssi toute suite décroissante est stationnaire.
\end{exo}



\begin{proposition} Soient $E$ et $F$ des ensembles ordonnés bien fondés.
Le produit $E\times F$ muni de l'ordre lexicographique est bien fondé.
\end{proposition}
\begin{proof}
Soit $X \subseteq E\times F$ une partie non vide.
Soit $A = \{e \in E\:\mid\: \exists f\in F, (e,f)\in X\}$ la projection de $X$ sur $E$, c'est-à-dire l'ensemble des premières coordonnées des éléments de $X$.
La partie $A\subseteq E$ est non vide, donc elle contient un élément minimal $m_1$.
Soit maintenant $B = \{f\in F\:\mid\: (m_1,f) \in X\}$.
La partie $B \subseteq F$ est non vide, elle contient un élément minimal $m_2$.

Alors, l'élément $(m_1,m_2)$ est un élément minimal de $X$ par construction : si un élément $(x,y) \in X$ vérifiait en effet $(x,y) <_{lex} (m_1,m_2)$, on aurait soit $x < m_1$, soit $x=m_1$ et $y<m_2$, ce qui est impossible par construction dans les deux cas.
\end{proof}

%Exemple d'ensemble pas bien fondé : mots sur un alphabet, avec l'ordre lexicographique.
%Même si sur l'alphabet c'est bien fondé, on a 
% $ab > aab > aaab > \cdots$
%Par contre l'ordre prefixe est bien fondé.


\begin{theoreme}[Principe de récurrence généralisée / récurrence bien fondée]
Soit $(E,\leq)$ un ensemble bien fondé et soir $<$ l'ordre strict associé.
Soit $\mathcal P(x)$ un prédicat\footnote{Une assertion à paramètre $x\in E$, autrement dit une fonction de $E$ dans les booléens.} sur $E$. 

Supposons que : 
\[ \forall x, (\forall y< x, P(y)) \implies P(x)\]
Alors, $\forall x\in E, P(x)$.
\end{theoreme}



\begin{proof}
Supposons $\forall x, (\forall y< x, P(y)) \implies P(x)$, assertion que l'on notera $(H)$.
Montrons $\forall x\in E, P(x)$.
Supposons par l'absurde que  $\forall x\in E, P(x)$ soit fausse.
La partie $A := \{x\in E, non(P(x))\}$ est alors non vide.
Elle possède donc un élément minimal, que l'on note $m$.
Par définition, $\forall y\in E, y< m \implies y\not\in A$, c'est-à-dire $\forall y\in E, y< m \implies P(m)$.
D'après $(H)$, on en déduit $P(m)$, contradiction.
\end{proof}

\begin{remarque}
L'hypothèse $\forall x, (\forall y< x, P(y)) \implies P(x)$ ressemble à une hypothèse d'hérédité de type récurrence forte, mais elle contient aussi l'initialisation (ou : les initialisations), car elle implique que pour tout $x$ minimal dans $E$, on a $P(x)$. En effet, il n'y a alors aucun élément $y<x$ donc la prémisse $\forall y< x, P(y)$ est vraie, donc par implication, $P(x)$ est vraie.
\end{remarque}


\section{Application : terminaisons d'algorithmes}

Rappel : la terminaison d'un algorithme n'est pas toujours facile à déterminer.
L'exemple le plus célèbre est  la conjecture de Syracuse, ou problème de Collatz : il n'est pas connu si la fonction Python ci-dessous termine pour toute entrée $n\in\N$ :

\begin{lstlisting}[language=Python]
def Syracuse(n):
	while n != 1:
		if n % 2 == 0:
			n = n / 2
		else:
			n = 3*n+1
	return 0 
\end{lstlisting}

Même lorsque l'on sait que l'algorithme termine, rédiger une preuve de la terminaison peut être plus ou moins long.

Dans les cas simples, comme le calcul récursif d'une factorielle, l'entrée est un entier positif qui diminue à chaque appel récursif, ce qui implique qu'il n'y a qu'un nombre fini d'appels successifs.

Mais parfois, il n'y a pas de choix clair d'entier positif qui diminue strictement, comme par exemple dans la fonction suivante qui prend en entrée un couple d'entiers strictement positifs : 
\begin{lstlisting}[language=Python]
def pgcd(a: int, b: int) -> int:
	# prend des entiers strictement positifs en entree
	if a == b:
		return a
	if a > b:
		return pgcd(a-b,b)
	else:
		return pgcd(a,b-a)
\end{lstlisting}

À chaque appel récursif, $a$ ne diminue pas forcément strictement, ni $b$.

Dans ce genre de situation, une technique de rédaction possible consiste à utiliser la notion d'ensemble ordonné bien fondé.

\begin{enumerate}
\item Dans le cas itératif, on va identifier un variant de boucle à valeur dans un ensemble bien fondé, qui décroit strictement à chaque tour de boucle. Ceci implique que la boucle ne tourne qu'un nombre fini de fois.
\item Dans le cas récursif, on va montrer que l'entrée appartient à un ensemble  bien fondé et qu'à chaque appel récursif,  l'entrée diminue strictement. La profondeur des appels récursifs est donc finie.
\end{enumerate}




Dans l'exemple de calcul récursif de pgcd, le couple $(a,b)$ est un couple d'entiers strictement positifs et s'il y a un appel récursif, il se fait sur un couple $(a',b')$ d'entiers strictement positifs qui est strictement inférieur à $(a,b)$ pour l'ordre lexicographique. 
Comme cet ordre est bien fondé, il n'existe donc pas de suites infinies strictement décroissantes pour cet ordre et donc l'algorithme termine.

%\begin{exemple}[Échauffement]
%La fonction suivante prend deux entiers naturels en entrée. 
%Prouver sa terminaison et dire ce qu'elle fait.
%\begin{lstlisting}[language=Python]
%def f(m,n):
%	# input : m, n entiers naturels
%	if n == 0:
%		return m
%	if m == 0:
%		return 0
%	return f(m-1,n-1)
%\end{lstlisting}
%\end{exemple}

\begin{exemple}
On suppose que l'on a une fonction \texttt{isPrime(n)} prenant  un entier et renvoyant un booléen, et une fonction \texttt{getNonTrivialDivisor(n)} prenant un nombre composé et renvoyant un diviseur non trivial de ce nombre.

Rédiger la preuve de terminaison de la fonction suivante, prenant en entrée un entier strictement positif. On utilisera la divisibilité sur $\N\setminus\{0,1\}$.
\begin{lstlisting}[language=Python]
def surprise(n):
	# input : un entier >=2 
	if isPrime(n):
		return 1
	a = getNonTrivialDivisor(n)
	return surprise(a) + surprise(n/a)
\end{lstlisting}
(Remarque : on pourrait être tenté de rédiger par récurrence forte sur $\N$, mais dans ce cas, il faut traiter à part le cas des nombres premiers. Il est plus élégant d'utiliser l'ordre de divisibilité et la récurrence bien fondée associée à cet ordre.)

\end{exemple}


\begin{exemple}% vu sur https://math.univ-lyon1.fr/irem/IMG/pdf/terminaison_avec_solutions.pdf
On suppose que l'on a une fonction \texttt{rand()} qui renvoie un nombre naturel (pseudo)aléatoire, non borné a priori.
Montrer la terminaison de la fonction suivante, qui prend deux nombres naturels en entrée. On utilisera l'ordre lexicographique.
\begin{lstlisting}[language=Python]
# rand() est une fonction retournant un nombre naturel aleatoire
def f(n,p):
	if p != 0:
		return f(n,p-1)
	elif n != 0:
		return f(n-1,rand())
	else:
		return 0 
\end{lstlisting}
(Remarque : si la fonction \texttt{rand()} était a priori bornée par un entier $M$, on pourrait montrer la terminaison de \texttt{f(n,p)} en majorant\footnote{On peut majorer le nombre d'appels récursifs par $p + (n-1)M$.} a priori le nombre d'appels récursifs en fonction de $n$, $p$ et $M$.
Mais dans l'énoncé donné, on ne peut pas procéder ainsi, et on ne peut pas majorer a priori le nombre d'appels récursifs. Néanmoins, on peut montrer que la fonction termine toujours.)
\end{exemple}

\paragraph{À retenir}
Il existe donc des algorithmes dont on arrive à montrer la terminaison tout en étant incapable de majorer le temps de terminaison en fonction de l'entrée.
Cette situation est nouvelle par rapport aux situation vues précédemment en cours, dans lesquelles on pouvait en général majorer le nombre d'appels récursifs ou de tours de boucle qui allaient être effectués sur une entrée donnée.

\begin{exemple}(Fonction d'Ackermann-Péter)
Prouver la terminaison de la fonction suivante, qui prend en entrée deux entiers naturels, en utilisant l'ordre lexicographique.
\begin{lstlisting}[language=Python]
def AP(m,n):
	# input : m, n entiers naturels
	if m == 0:
		return n+1
	if n == 0:
		return AP(m-1,1)
	return AP(m-1,AP(m,n-1))
\end{lstlisting}
\end{exemple}

% Fonction 91 ? Non garder pour l'examen


\begin{exemple}[Petites suites de Goodstein]% vu sur https://math.univ-lyon1.fr/irem/IMG/pdf/terminaison_avec_solutions.pdf
On considère l'algorithme suivant, qui prend un entier naturel $N$ en entrée :
\begin{enumerate}
\item on l'écrit en base deux, puis on soustrait $1$;
\item on interprète cette écriture comme une écriture en base trois, puis on soustrait $1$;
\item on interprète cette écriture comme une écriture en base quatre, puis on soustrait $1$;
\item etc.
\end{enumerate}
Cette suite d'opérations finit toujours par aboutir  l'entier nul !
Par exemple si on commence avec $N=6$:
\begin{enumerate}
\item on l'écrit en base deux ($\overline{110}^2$), on enlève un  ($\overline{101}^2$), 
\item on interprète l'écriture en base trois ($\overline{101}^3$), on enlève un ($\overline{100}^3$),
\item on interprète l'écriture en base quatre ($\overline{100}^4$), on enlève un ($\overline{33}^4$),
\item etc.
\end{enumerate} 
Les valeurs successives de l'entier sont alors $6$, $5$, $10$, $9$, $16$, $15$, $18$, $17$, \dots

L'observation cruciale est la suivante : la longueur des écritures (en base deux, trois, quatre etc) va décroitre au sens large.
Les chiffres apparaissant dans ces écritures vont eux potentiellement augmenter, mais on peut raisonner sur $\N^r$ muni de l'ordre lexicographique, où $r$ est le nombre de chiffres en base deux de la valeur initiale.
\end{exemple}

\section{Ouverture}

La récurrence bien fondée est utilisée par de nombreux logiciels de vérification de preuves ou de code informatique. Lorsque ces techniques ne suffisent plus, ils font appel à d'autres méthodes plus récentes.

Lectures possibles pour approfondir: 
\begin{enumerate}
\item \url{https://en.wikipedia.org/wiki/Well-quasi-ordering}
\item Well quasi-orders for algorithms, (cours de L3 info) \url{https://wikimpri.dptinfo.ens-cachan.fr/lib/exe/fetch.php?media=cours:upload:poly-2-9-1v02oct2017.pdf}
\end{enumerate}


\end{document}

\begin{exemple}
\begin{lstlisting}[language=Python]

\end{lstlisting}
\end{exemple}

(idée TP : calculer la complexité en moyenne sur des entiers bornés par N ? regarder https://alainbusser.frama.io/NSI-IREMI-974/euclide.html )

- petit exo (garder pour exam ?)(terminaison facile, correction aussi, virer dans autre fichier)
Q : que fait la fonction suivante ?
```python
def mystere(a: int) -> int:
	if a < 0:
		a = -a
	c = 0
	b = a
	while b != 0:
		c += a
		b -= 1
	return c

```


% https://www.lip6.fr/Christine.Tasson/doc/cours/M1sem/TD1.pdf
Exercices : ordre lexicographique, ordre multiset et application à la terminaison du parcours des graphes ?


Non vu mais demander aux étudiants de se renseigner, potentiel exo d'examen ?
---

Treillis (deux éléments ont une borne sup et une borne inf), treillis complets, fonditions monotones, théorème de points fixes

point d'entrée : http://lim.univ-reunion.fr/staff/fred/Stages-Ters-Theses/Richard-Fontaine/Equivalence-et-Ordres.pdf
(ou option info)
